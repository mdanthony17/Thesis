% This is the abstract of my dissertation.

\pagestyle{empty} % No page number in entire abstract
\begin{center}
  ABSTRACT

    Understanding Low-Energy Nuclear Recoils in Liquid Xenon for Dark Matter Searches and the First Results of XENON1T.

    Matthew Duschl Anthony
\end{center}

An abundance of cosmological evidence suggests that cold dark matter exists and makes up 83\% of the matter in the universe.  At the same time, this dark matter has eluded direct detection and its identity remains a mystery.  Many large international collaborations are actively searching for dark matter through its potential annihilation in high-density regions of the universe, its creation in particle accelerators, and its interaction with standard model particles in low-background detectors.

One of the most promising dark matter candidates is the weakly interacting massive particle (WIMP) which falls naturally out of extensions of the Standard Model.  A variety of detectors have been employed in the search for WIMPs, which are expected to scatter with atomic nuclei, yet none have been more successful than dual-phase liquid xenon time projection chambers (TPCs).  The first ton-scale liquid xenon TPC, XENON1T, began operating in 2016 and with only 34.2 days of data has set the most strict limits on the WIMP-nucleon interaction cross sections for WIMP masses above 10 GeV/$\textrm{c}^2$, with a minimum of \num{7.7e-47} $\textrm{cm}^2$ for 35 GeV/$\textrm{c}^2$ WIMPs.

One of the major keys to success for liquid xenon TPCs is our understanding of interactions in the medium through myriad measurements.  Given that the expected WIMP scattering rate increases with decreasing interaction energy, there has been more focus in recent years in pushing our understanding of interactions in liquid xenon to lower energies.  Additionally, as liquid xenon TPCs operate with a large electric field in the medium, an effort has been made to understand how the signal response of xenon changes as a function of the applied electric field.      
   
In this thesis, I describe the details of XENON1T, its calibration and characterization, with a special emphasis on the electronic and nuclear recoil calibrations, and the inaugural WIMP search of XENON1T.    I then discuss a dedicated measurement, made in the calibration-optimized liquid xenon TPC neriX, of the signal response of low energy nuclear recoils in liquid xenon at electric fields relevant to the dark matter search.  The measurements of signal response in XENON1T and neriX were performed using an analysis framework that I developed to allow a more sophisticated examination of recoil responses using GPU-accelerated simulations.


