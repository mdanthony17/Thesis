%This is the second chapter of the dissertation

%The following command starts your chapter. If you want different titles used in your ToC and at the top of the page throughout the chapter, you can specify those values here. Since Columbia doesn't want extra information in the headers and footers, the "Top of Page Title" value won't actually appear.

\pagestyle{cu}
\graphicspath{{./Chapter2/images/}}

\chapter[Liquid Xenon and Dual-Phase TPCs][Liquid Xenon and Dual-Phase TPCs]{Liquid Xenon and Dual-Phase TPCs}

This chapter will focus on the liquid xenon as a detector medium.  In \secref{sec:lxe_chem_properties} we will discuss the chemical properties of liquid xenon along with some of the benefits and considerations of these properties.  In \secref{sec:lxe_er} we will discuss the production of observable light and charge from electron recoils while in \secref{sec:lxe_nr} we will discuss observable production from elastic nuclear recoils.  Finally, in \secref{sec:lxe_tpc}, we will discuss how these observables are detected in dual-phase xenon time projection chambers.

\section{Chemical Properties}
\label{sec:lxe_chem_properties}

Xenon, with an atomic number of 54, is a noble gas meaning that it has a full valence electron shell.  Because of the full valence shell, xenon is very unlikely to interact chemically with other elements and molecules.  Xenon is also the heaviest noble gas that is, for practical purposes, naturally non-radioactive.  \ce{^{136}Xe}, with a natural abundance of 8.857\%, has been show to undergo double beta decay with a half-life of $2.165 \cdot 10^{21}$ years so strictly speaking natural xenon is radioactive although this process is extremely rare and has no relevance for even low background dark matter experiments \cite{albert2014improved}.

While natural xenon is not radioactive, it is actually possible to excite xenon nuclei such that they decay and emit gamma rays.  None of these excited states have very long lifetimes that would cause issues for low background experiments but two of these neutron activated states (\ce{^{131m}Xe} and \ce{^{129m}Xe} which decay emitting 164 keV and 236 keV photons, respectively) have half-lives on the order of ten days.  This half-life could potentially be very useful in the calibration of large detectors since over this period of time the excited states would be approximately uniformly distributed inside of the detector \cite{ni2007preparation}.

Xenon is extracted from the atmosphere as a byproduct of the separation of oxygen and nitrogen.  Once the oxygen is separated, it will contain trace amounts of krypton and xenon that can be separated out further by distillation or adsorption.  The xenon that is purchased commercially typically will have a final Kr/Xe ratio of $\sim 10^{-6} - 10^{-9} \frac{\textrm{mol}}{\textrm{mol}}$.  Natural krypton is not radioactive on a relevant time scale but \ce{^{85}Kr}, which is released into the atmosphere via nuclear fuel reprocessing and nuclear weapons tests, beta decays with a mean energy of 251 keV and with a half-life of roughly 10.8 years \cite{abe2009distillation}.  So while natural xenon is not radioactive, the process of extracting xenon from the atmosphere does leave a radioactive isotope that could be a potential source of background for dark matter experiments.  Significant effort has gone into reducing the Kr/Xe levels to reduce this background as much as possible.  In XENON1T, the lowest level to date was achieved with a natural krypton to xenon ratio of less than 200 ppq (1 ppq = $10^{-15} \frac{\textrm{mol}}{\textrm{mol}}$) \cite{aprile2017removing}.



\begin{table}[t]
\centering
\label{tab:xe_abundance}
\def\arraystretch{1.3}
\begin{tabular}{ccccc}
\hline
Isotope & Abundance & Spin & Half-life & Decay Mode  \\ \hline
\ce{^{124}Xe} & 0.095\% & 0 & $ > 1.6 \cdot 10^{14}$ y & $2 \nu \beta^+ \beta^+$ \cref{fn:predicted_decay} \\ %\hline
\ce{^{126}Xe}&  0.089\% & 0 & $ 4.7 - 12 \cdot 10^{25}$ y & $2 \nu \beta^- \beta^-$ \cref{fn:predicted_decay} \\ %\hline
\ce{^{128}Xe}&  1.910\% & 0 & Stable & - \\ %\hline
\ce{^{129}Xe}&  16.400\% & $\sfrac{1}{2}$ & Stable & - \\ %\hline
\ce{^{130}Xe}&  4.071\% & 0 & Stable & - \\ %\hline
\ce{^{131}Xe}&  21.232\% & $\sfrac{3}{2}$ & Stable & - \\ %\hline
\ce{^{132}Xe}&  26.909\% & 0 & Stable & - \\ %\hline
\ce{^{134}Xe} & 10.436\% & 0 & $ > 5.8 \cdot 10^{22}$ y & $2 \nu \beta^- \beta^-$ \cref{fn:predicted_decay} \\ %\hline
\ce{^{136}Xe}&  8.857\% & 0 & $2.2 \cdot 10^{21}$ y & $2 \nu \beta^- \beta^-$ \\ %\hline
\end{tabular}
\caption{Abundances, half-lives, and decay modes of various xenon isotopes.  Note that \ce{^{136}Xe} is the only isotope whose decay has been measured.  Half-life data: \cite{barros2014double}.}
\end{table}
\stepcounter{footnote}



Dual-phase xenon experiments typically operate at roughly 2--3 atm, which translates to a boiling point of roughly 180 K ($-93.2^{\circ}$ C).  The density of liquid xenon (LXe) at this temperature is roughly 2.84 $\sfrac{\textrm{g}}{\textrm{cm}^3}$ which is significantly higher than all of the other noble elements, with the exception of radon \cite{rankin2009crc}.  The high density of LXe is partly responsible for its high electronic stopping power, which will be discussed further in the next section.

\begin{figure}[t]
	\centering
	\includegraphics[width=0.7\textwidth]{xe_pt_diagram}
	\caption{The phase diagram for xenon.  Dual-phase xenon TPCs typically operate in the range of 2--3 atm.}
	\label{fig:xe_phase_diagram}
\end{figure}



\section{Energy Deposition of Charged Particles in Liquid Xenon}
\label{sec:energy_deposition}

Both nuclear and electronic recoils, which will be discussed in the following sections, ultimately result in a charged particles traversing the LXe - in the case of an electronic recoil the resulting charged particle is an electron and in the case of nuclear recoils it is the xenon nucleus.  Given the high density and atomic number of xenon, the electronic stopping power is large for both electrons and xenon ions ($\sim 1-30 \sfrac{\textrm{keV}}{\mu \textrm{m}}$).  This means that the tracks of low energy electronic and nuclear recoils will be very small and approximately point-like \cite{aprile2006simultaneous}.

In liquid xenon (and other noble liquids), scintillation light is produced via the excitation of atomic electrons and the ionization and subsequent recombination of free electrons and ions.  The excitation scintillation process is shown in \eqnref{eqn:exciton_production} and the ionization scintillation process is shown in \eqnref{eqn:ionization_production}.  


% foot note for table
\footnotetext{\label{fn:predicted_decay}This decay is predicted but has not yet been observed.}

\begin{equation}
        \label{eqn:exciton_production}
        \begin{gathered}
                \textrm{Xe*} + \textrm{Xe} + \textrm{Xe} \rightarrow \textrm{Xe*}_2 + \textrm{Xe}, \\
                \textrm{Xe*}_2 \rightarrow 2\textrm{Xe} + h \nu
        \end{gathered}
\end{equation}


\begin{equation}
        \label{eqn:ionization_production}
        \begin{gathered}
                \textrm{Xe}^+ + \textrm{Xe} \rightarrow \textrm{Xe}_2^+, \\
                \textrm{Xe}_2^+ + e^- \rightarrow \textrm{Xe**} + \textrm{Xe}, \\
                \textrm{Xe**} \rightarrow \textrm{Xe*} + \textrm{Xe}, \\
                \textrm{Xe*} + \textrm{Xe} + \textrm{Xe} \rightarrow \textrm{Xe*}_2 + \textrm{Xe}, \\
                \textrm{Xe*}_2 \rightarrow 2\textrm{Xe} + h \nu
        \end{gathered}
\end{equation}

The excitation process proceeds when an an atomic electron in xenon is excited, which is referred to as an \textit{exciton}, and forms a dimer with another xenon atom, which is called an \textit{excimer}.  This excited excimer can be formed in either the singlet state (spin of excited electron anti-parallel to electron originally sharing state) or triplet state (spin of excited electron parallel to electron originally sharing state).  The excimers in the singlet and triplet states each have their own characteristic lifetimes (roughly 4 ns and 22 ns)\footnote{In xenon the difference in lifetimes of the singlet and triplet states is fairly small but for argon the singlet lifetime is 7 ns while the triplet lifetime is 1.6 $\mu$s!} and decay into xenon atoms and a 178 nm photon (the photon falls in UV portion of the spectrum) \cite{hitachi1983effect, doke2002absolute}.  

The ionization process begins when a charged particle ionizes a xenon atom, leaving singly-ionized xenon and a free electron.  The singly-ionized xenon atom can then form an ionized dimer and subsequent excited xenon state.  This excited xenon state leads to an excimer through non-radiative heat loss.  The excimer produces scintillation light in the manner described above.

% mention charge signal
Implicit in the ionization process outlined above is the assumption that the electron freed during ionization recombines with the singly-ionized dimer.  However, in the presence of an electric field, this recombination can be reduced such that a charge signal can also be read out in addition to the scintillation signal.  Incomplete recombination can also occur at zero electric field and these electrons are called \textit{escape electrons} (although you cannot extract the charge signal without an applied electric field) \cite{doke2002absolute}.


\begin{figure}[t]
	\centering
	\includegraphics[width=0.7\textwidth]{observables_diagram}
	\caption{}
	\label{fig:diagram_energy_deposition}
\end{figure}

% important to note xenon ions lose energy via heat too...
% pg 4 of Lindhard paper
It is important to note that while these electronic excitation and ionization mechanisms are dominant for electronic recoils, the energy deposition for nuclear recoils is split between these and atomic motion.  This distinction is extremely important - the energy given to electrons in a recoil cannot cause atomic motion however atomic motion, if sufficiently slow, will not be able to cause excitation or ionization in other atoms and hence some energy is lost.  This effect was first discussed by Lindhard in 1963 \cite{lindhard1963integral} and the effort to quantify this effect continues today and in this work.  This effect will henceforth be referred to as nuclear quenching.

A second form of quenching has been observed in high linear energy transfer (LET) interactions, specifically with $\alpha$ scatters in xenon (which will not be discussed in detail) and high energy nuclear recoils.  This quenching is called biexcitonic quenching and is the result of two excitons colliding to produce an electron-ion pair as shown in \eqnref{eqn:biexcitonic_quenching}.

\begin{equation}
        \label{eqn:biexcitonic_quenching} 
        \textrm{Xe*} + \textrm{Xe*} \rightarrow \textrm{Xe} + \textrm{Xe}^+ + e^-
\end{equation}

Since this form of electronic quenching requires the collision of two excitons, it is expected that the track density ultimately determines the level of quenching \cite{hitachi2005properties}.

A diagram showing all of the mentioned energy deposition methods for charged particles is shown in \figref{fig:diagram_energy_deposition}.



\section{Electronic Recoils in Liquid Xenon}
\label{sec:lxe_er}

In this section, we will discuss the sources of electronic recoils in liquid xenon, their properties, and how they result in detectable observables.  For dual-phase LXe TPCs, which we will focus on in more detail later, searching for ``standard'' WIMPs, electronic recoils constitute the background.  With a precise understanding of what causes electronic recoils and how they interact in LXe, we can better discriminate between electronic recoils and potential signals that are expected to interact via nuclear recoils.  Additionally, if WIMPs do interact with atomic electrons rather than the nucleus, a precise understanding of the electronic recoil background would be crucial for a discovery.  

\subsection{Sources of Electronic Recoils}

There are two main sources of energetic electrons in liquid xenon: (1) beta decays from contaminants inside of the detector and (2) photons interacting through matter via photoelectric absorption, Compton scattering, or pair production.  In either case, the resulting energetic electron creates a track through the xenon, mainly losing its energy from inelastic collisions with atomic electrons.   In standard WIMP hypotheses, WIMPs are expected to interact with the atomic nucleus, however there are certain theories of WIMPs that allow interactions between a WIMP and atomic electrons that would result in an electronic recoil. 

\subsubsection{Beta Decays}

While there are both $\beta^-$ and $\beta^+$ decays, we will focus on $\beta^-$ decays since they are relevant to WIMP searches.  $\beta^-$ decay is a radioactive decay in which a neutron is converted to a proton inside of the nucleus and a subsequent electron and anti-electron neutrino are emitted.  This type of decay is made possible by the weak force which allows a quark to change its type via a W boson and an electron and anti-neutrino (positron and neutrino) pair \cite{cottingham1987introduction}.

While the maximum energy of the energetic electron in the decay is fixed, because an anti-neutrino is also emitted in $\beta^-$ decay, the energy spectra of the electron is continuous.  This continuous energy spectrum is what makes long-lived $\beta^-$ emitters very dangerous potential sources of background - they can, with non-negligible probabilities, produce electrons with energies of interest for WIMP detection ($\lesssim 30$ keV).  The energy spectrum for \ce{^{83}Kr} is shown in \figref{fig:kr85_beta_decay}.

\begin{figure}[t]
	\centering
	\includegraphics[width=0.7\textwidth]{kr85_beta_rates}
	\caption{The kinetic energy spectrum of electrons resulting from the $\beta^-$ decay of \ce{^{85}Kr} \cite{mantel1972beta}.  Note that roughly 6.5\% of decays are below 30 keV (shaded red region) which puts them inside the energy region of interest of WIMP searches.}
	\label{fig:kr85_beta_decay}
\end{figure}

% show multiple energy spectra of the beta decays from 
%https://ac.els-cdn.com/0020708X7290107X/1-s2.0-0020708X7290107X-main.pdf?_tid=259893f2-a843-11e7-9dfb-00000aacb35f&acdnat=1507039388_aef3096471a4defbd8b1c8a8198864c6

In liquid xenon based detectors, the two biggest sources of background beta decays are from \ce{^{85}Kr} and \ce{^{214}Pb}, which comes from the \ce{^{222}Rn} decay chain \cite{aprile2017first}.  Even though certain atoms that $\beta^-$ decay prove to be a background that must be carefully reduced, others have proven to be extremely useful for detector calibrations.  \ce{^{212}Pb}, from the decay chain of \ce{^{220}Rn},  has proven useful for calibrations since approximately 10\% of electrons have an energy less than 30 keV (the maximum energy is 570 keV) \cite{aprile2017radon}.  Perhaps even more exciting for the low energy calibrations of electronic recoils is the use of tritium, which has a maximum energy of only 18.6 keV \cite{akerib2016tritium, aprile2017tritium}! \footnote{Molecular tritium ($\textrm{T}_2$) cannot be used because it adsorbs to surfaces very easily and the half-life of $\textrm{T}_2$ is 12.3 years.  Instead, tritiated methane ($\textrm{CH}_3\textrm{T}$) is used since this will not adsorb and can be easily removed.}  

\subsubsection{Photons}

Another source of electronic recoils in LXe comes from photons.  Photons, via photoelectric absorption, Compton scattering, or pair production, can create energetic electrons inside of the detector.  While pair production is not relevant in the energy range of interest, photoelectric absorption is one of the most tried and tested calibration tools for LXe (and other detectors) and electrons from Compton scatters can make up part of the background in WIMP searches since the energy of the electron can be arbitrarily low.

Photoelectric absorption is the process by which a photon is absorbed by an atom from which an electron is subsequently ejected (typically from the K shell).  This implies that the energy of the ejected electron is equal to the energy of the photon minus the binding energy.  However, the newly ionized atom will have a free electron bind with it, usually on a very short time scale, and an X-ray or auger electron will be emitted \cite{knoll2010radiation}.  Therefore the energy detected from photoelectric absorption will be very close to the initial energy of the photon.  Photoelectric absorption is the dominant mode of interaction up to a few hundred keV in most media, including xenon as can be seen in \figref{fig:photon_attenuation}.  

\begin{figure}[t]
	\centering
	\includegraphics[width=0.95\textwidth]{photon_attenuation}
	\caption{The mass attenuation coefficient and the attenuation for photons of different energies in liquid xenon \cite{berger8coll}.}
	\label{fig:photon_attenuation}
\end{figure}

Compton scattering is the process by which a photon interacts with an atomic electron resulting in the deflection of the photon at a specific angle and a transfer of energy to the electron.  The angle of the scattering completely describes the energy transferred to the electron.  Compton scattering is the dominant mode of interaction from a few hundred keV to a few MeV in most media, including xenon as can be seen in \figref{fig:photon_attenuation}.

\figref{fig:photon_attenuation} shows the mass attenuation coefficient of photons in LXe and the individual contributions of each process.  Because of xenon's high atomic number, all processes have very high attenuation coefficients.  This is valuable for background reduction since low energy photons are absorbed at the very edge of the detector (since their attenuation length is $<$ 1 cm) although it does make calibration with external gamma ray sources very difficult for large detectors.\footnote{This is the reason why many large scale LXe detectors are calibrated using internal sources now such as the beta emitters mentioned earlier and metastable activated xenon.}  Compton scatters have an attenuation length on the order of several centimeters which means that they will contribute to the background of LXe detectors at some level.
 

\subsubsection{Neutrons}

Neutrons can interact in liquid xenon mainly through three mechanisms: radiative absorption and inelastic scattering, which result in electronic interaction in the medium, and elastic scattering, which ultimately results in a nuclear recoil and will be discussed in \secref{sec:lxe_nr}.   The cross-sections of each of these mechanisms for xenon can be seen in \figref{fig:neutron_cross_section}.  Note that for almost all energies between 1 keV -- 10 MeV that elastic scattering is the dominant process.


\begin{figure}[t]
	\centering
	\includegraphics[width=0.95\textwidth]{neutron_cross_sections}
	\caption{ \cite{chadwick2011endf}.}
	\label{fig:neutron_cross_section}
\end{figure}


Radiative absorption is the absorption of neutrons by a nucleus.  The nucleus thus increases by one in mass number with atomic number staying the same.  Fortunately, since the isotopes of xenon that could be produced are not radioactive, with the exception of \ce{^{133}Xe} and \ce{^{135}Xe}, this process produces very little background.  \ce{^{133}Xe} and \ce{^{135}Xe} both result in short $\beta^-$ chains and will therefore result in electronic recoils inside of the detector.  With this said, for the neutron energies of background and calibrations in liquid xenon WIMP detectors, radiative absorption is largely irrelevant.

Inelastic scattering is the process by which a particle interacts with the atomic nucleus and kinetic energy is lost due to the excitation of the nucleus.  The excitation of the nucleus, also called \textit{activation}, is then followed by the nucleus decaying from this excited state back down to a stable state through the emission of a particle.  For xenon, there are two inelastic collisions of note:  an inelastic scattering with \ce{^{129}Xe} or \ce{^{131}Xe}.  A neutron scattering inelastically with \ce{^{129}Xe} can result in nucleus being in an excited state with a 0.96 ns half-life that decays into gamma ray at an energy of approximately 40 keV or in an excited metastable state with a half-life of 8.8 days that results in a 197 keV photon followed by a 40 keV photon (the 40 keV photon is from the same very short lived state that the metastable state decays into) \cite{timar2014nuclear}.  A neutron scattering inelastically with \ce{^{131}Xe} can result in the nucleus being in a metastable state with a half-life of 11.84 days that decays emitting a 164 keV photon \cite{khazov2006nuclear}.   While these process are not relevant for background considerations during a WIMP search, they are very useful when calibrating a detector since they each result in electronic recoils at a low and fixed energy.

There are three major sources of neutrons in dark matter experiments.  The first major source is from heavy elements in various detector components decaying via spontaneous fission resulting in neutrons with energies typically from 1 -- 10 MeV.  Neutrons also come from high-energy muons interacting with the rock and materials around the detector.  Finally, neutrons can be produced artificially using a neutron generator (typically either through a deuterium-deuterium reaction or deuterium-tritium reaction).  The first two sources of neutrons make up background in dark matter searches while the third source of neutrons is used to calibrate detectors (for both electronic and nuclear recoils).


\subsubsection{Neutrinos}

% https://arxiv.org/pdf/1512.07501.pdf
% largest source of neutrinos is the sun

Neutrinos can elastically scatter with electrons either via charged-current (exchange of W boson) or neutral-current (exchange of Z boson) interactions.  For electronic recoils, the main sources of neutrinos are from initial deuterium production and \ce{^7Be} reactions inside the sun (roughly 92\% and 7\% of the neutrino background, respectively) \cite{aprile2016physics}.  Like electronic recoils from beta decays, the kinetic energy of the recoiling electron will follow a spectrum where only very low energies ($\lesssim 30$ keV) are relevant.  Unlike other sources of electronic recoils, the solar neutrino background cannot be reduced.

% example recoil spectrum shown in: https://journals.aps.org/rmp/pdf/10.1103/RevModPhys.59.505

%franarin2016reducing
\begin{figure}[t]
	\centering
	\includegraphics[width=0.7\textwidth]{solar_neutrino_flux}
	\caption{Solar neutrino fluxes from different processes assuming the BS05(OP) standard solar model.  Image Courtesy: \cite{franarin2016reducing}.}
	\label{fig:solar_neutrino_flux}
\end{figure}

\subsection{Observables Production for Electronic Recoils}

In \secref{sec:energy_deposition}, we discussed the modes by which charged particles deposit energy in LXe.  We will now quantify these observables production methods for electronic recoils under the assumption of an applied electric field.

As mentioned in \secref{sec:energy_deposition}, electronic recoils result in either excitation or the creation of electron-ion pairs.  Assuming the recoils occur in the presence of an electric field, we do not need to be concerned about quenching with respect to escape electrons (since these can be extracted by the electric field and ultimately measured).  Additionally, electronic recoils have relatively sparse tracks (as can be seen by their low stopping power in liquid xenon) \cite{aprile2006simultaneous} so it is expected that biexcitonic quenching will not play a large role in observables production.

Since there are no major forms of quenching, we can completely separate the energy deposited in the electronic recoil into excitons and electron-ion pairs.  Typically the total number of quanta (excitons and electron-ion pairs) is used to describe this relationship - specifically, the average energy required to produce a single quanta.  For xenon, this value is $W = 13.7 \pm 0.2 \, \textrm{eV}$ \cite{dahl_thesis} and the relationship is given by \eqnref{eqn:w_value}.

\begin{equation}
        \label{eqn:w_value}
        \textrm{N}_q = \frac{\textrm{E}_{\textrm{ER}}}{W} = \textrm{N}_{\textrm{ex}} + \textrm{N}_{\textrm{ion}}
\end{equation}  

This relationship, while looking very simple, turns out to be extremely useful for calibrations in dual-phase xenon TPCs, as we will discuss in later chapters.  The breakdown of excitons to electron-ion pairs is simply described by the ratio of the two quantities such that we can define probabilities of a given quanta being an exciton or electron-ion pair.

\begin{equation}
        p_{\textrm{ion}} = \frac{1}{1 + \frac{\textrm{N}_{\textrm{ex}}}{\textrm{N}_{\textrm{ion}}}}, \, \, \, p_{\textrm{ex}} = 1 - p_{\textrm{ion}}
\end{equation}

The exciton-to-ion ratio, $\frac{\textrm{N}_{\textrm{ex}}}{\textrm{N}_{\textrm{ion}}}$, has been theoretically calculated to be 0.06 for sub-MeV electronic recoils \cite{takahashi1975average} however measurements and theoretical predictions have also suggested a value of $0.20 \pm 0.13$ \cite{doke2002absolute, aprile2007observation}.   

As mentioned previously, electron-ion pairs have a finite probability of recombining to form excitons and eventually producing a scintillation signal (as opposed to a charge signal).  While in the past this recombination probability was modelled using Birks' saturation law \cite{birks2013theory} for high track densities and the Thomas-Imel model \cite{thomas1987recombination} (which will be discussed in more detail for nuclear recoils) for low track densities, recently a great deal of work has gone into directly measuring recombination in liquid xenon and its potential fluctuations without the assumption of a model \cite{akerib2016tritium, aprile2017tritium}.  Recombination is simply inserted to the model of observables production as shown in \eqnref{eqn:recombination_er}.

\begin{equation}
        \label{eqn:recombination_er}
        \textrm{N}_{\textrm{ex}} \leftarrow \textrm{N}_{\textrm{ex}} + r \textrm{N}_{\textrm{ion}}, \, \, \, \textrm{N}_{\textrm{ion}} \leftarrow (1 - r) \textrm{N}_{\textrm{ion}}
\end{equation}



\section{Nuclear Recoils in Liquid Xenon}
\label{sec:lxe_nr}

It is expected that WIMPs could potentially dissipate energy in xenon via elastic nuclear recoils so understanding these type of interactions is of crucial important for WIMP direct detection experiments.  In this section, we will discuss the sources of nuclear recoils in liquid xenon based WIMP searches (besides potential WIMPs) and the observables production process for elastic nuclear recoils, which is substantially more complicated due to the nuclear and electronic quenching first mentioned in \secref{sec:energy_deposition}.

\subsection{Sources of Nuclear Recoils}

The two sources of nuclear recoils in liquid xenon based WIMP searches, besides potential WIMPs, are neutrons and neutrinos.   While neutrons are, as one would expect, the main background and calibration source in liquid xenon based WIMP searches, neutrinos are no longer negligible and, as detectors become more and more sensitive to lower cross-sections, will soon comprise an irreducible background of elastic nuclear recoils in detectors.  Understanding the sources of nuclear recoils in liquid xenon based WIMP direct detection experiments is crucially important since an underestimation of the background could lead to potential claims of a false WIMP signal since interactions would be indistinguishable on an event-by-event basis.

\subsubsection{Neutrons}

Electronic recoils from neutron scattering were discussed in \secref{sec:lxe_er} --- in this section we will focus on nuclear recoils from elastic scattering.  Elastic scattering is the process by which a particle interacts with the atomic nucleus and kinetic energy is conserved.  The recoiling nucleus then deposits its energy in the medium which can ultimately be detected.  Particles scattering elastically with nuclei is also called a nuclear recoil.  Of course, this process is not unique to neutrons but is the main mode of interaction for many massive particles (and hopefully WIMPs).  

Each of the sources of neutrons mentioned in \secref{sec:lxe_er}, spontaneous fission of heavy materials, high-energy muons, and artificially generated muons, can also result in nuclear recoils.




\subsubsection{Neutrinos}

Neutrinos can interact with both electrons, as discussed in \secref{sec:lxe_er}, and atomic nuclei, via coherent neutrino-nucleon scattering (CNNS).  The maximum energy of a recoiling nucleus is given by $E_{\textrm{r}}^{\textrm{max}} = \frac{2 E_{\nu}^2}{m_N + 2 E_{\nu}}$, where $m_N$ is the mass of the nucleus and $E_{\nu}$ is the energy of the neutrino.  This implies that neutrinos must have energies on the order of 10 MeV to cause nuclear recoils on the order of 1 keV.  Therefore, high energy neutrino sources like \ce{^{8}B} in the sun as well as neutrinos from supernovae and the atmosphere will contribute the most to the CNNS background in dark matter experiments.


\subsection{Observables Production for Nuclear Recoils}

We will now discuss the details of the observables production process for nuclear recoils that was generally outlined in \secref{sec:energy_deposition}.  Like electronic recoils, nuclear recoils can lead to the excitation or ionization of other xenon atoms.  However, unlike energetic electrons in liquid xenon, recoiling xenon atoms will also interact with other xenon nuclei.  This distinction is extremely important since energy can effectively be ``lost'' if the energy transferred during a collision is too low to cause excitation or ionization.  

Lindhard proposed a theory to describe this nuclear queching in \citeref{lindhard1963integral}.  To describe the quenching of signals due to atomic motion, it is standard to work with the dimensionless energy given in \eqnref{eqn:dimensionless_energy}.

\begin{equation}
        \label{eqn:dimensionless_energy}
        \epsilon = 11.5 \left( \frac{E}{\textrm{keV}} \right) Z^{\sfrac{-7}{3}}
\end{equation}

Lindhard showed that at low velocities ($v < v_F$) the stopping power of a heavy ion in a medium is approximately given $S_e = k \epsilon^{\sfrac{1}{2}}$, where $k$ is a proportionality constant.  Under the same assumptions, it can be shown that $k = 0.133 Z^{\sfrac{2}{3}} A^{-\sfrac{1}{2}}$, which would give $k \approx 0.165$ for xenon, although in his original paper Lindhard names this proportionality factor as the largest source of uncertainty.  Shown in \eqnref{eqn:lindhard_electronic} is Lindhard's semi-empirical numerical solution for the fraction of the total energy that goes to electronic interactions for recoiling atoms.

\begin{equation}
        \label{eqn:lindhard_electronic}
        L(\epsilon) = \frac{k g(\epsilon)}{1 + k g(\epsilon)}, \, \, \, g(\epsilon) = 3 \epsilon^{0.15} + 0.7 \epsilon^{0.6} + \epsilon
\end{equation}

Note that $g(\epsilon)$ is not derived from first principles but is a fit to Lindhard's numerical solution from $\epsilon = 0.001 - 100$ (roughly 1 keV -- 100 MeV nuclear recoils for xenon).

Similar to observables production in electronic recoils, we assume that all energy that goes towards electronic interactions is converted into excitons and ions by way of the W value as is shown in \eqnref{eqn:quanta_nr}.

\begin{equation}
        \label{eqn:quanta_nr}
        \textrm{N}_q = \frac{L(E) E_{\textrm{NR}}}{W} = \textrm{N}_{\textrm{ex}} + \textrm{N}_{\textrm{ion}}
\end{equation}

As with electronic recoils, the split into excitons and ions can be defined by a single parameter, $\frac{\textrm{N}_{\textrm{ex}}}{\textrm{N}_{\textrm{ion}}}$.

\begin{equation}
        p_{\textrm{ion}} = \frac{1}{1 + \frac{\textrm{N}_{\textrm{ex}}}{\textrm{N}_{\textrm{ion}}}}, \, \, \, p_{\textrm{ex}} = 1 - p_{\textrm{ion}}
\end{equation}

Unlike electronic recoils, however, it is expected that $\frac{\textrm{N}_{\textrm{ex}}}{\textrm{N}_{\textrm{ion}}} \approx 1$ for nuclear recoils \cite{angle2011search, lenardo2015global}.

SRIM (Stopping and Range of Ions in Matter) simulations show that nuclear recoils, unlike electronic recoils, lose the majority of their energy in a large number of secondary tracks implying a very low track density.  At low track densities and with applied electric fields we can use the Thomas-Imel recombination model to describe the recombination of electrons and ions into excitons \eqnref{eqn:ionization_production}.    



% use following for biexcitonic quenching:
% https://ac.els-cdn.com/S0927650505000964/1-s2.0-S0927650505000964-main.pdf?_tid=e866eaac-a918-11e7-974f-00000aab0f6b&acdnat=1507131190_5bd95affa2fae24418f54ce96721ae8a
% https://arxiv.org/pdf/0712.2470.pdf
% https://ac.els-cdn.com/S0927650511001289/1-s2.0-S0927650511001289-main.pdf?_tid=baf2e1c0-a922-11e7-a7e2-00000aacb35d&acdnat=1507135409_d72980a38a6fbe81c7b4585e9dae6c97


\section{Dual-Phase Time Projection Chambers}
\label{sec:lxe_tpc}


\subsection{Observables Production}